Hoxe en día un sistema operativo é un software moi complexo que permite
que o hardware sexa \emph{transparente} para un usuario, é dicir, non
hai que se preocupar da marca do escáner se queremos obter unha imaxe na
pantalla. Salvo cando falamos a moi baixo nivel (con instrucións en
código máquina) o resto dos programas poden executarse para o hardware
de calquera máquina na que estea instalado o mesmo sistema operativo. O
sistema operativo é xa que logo o organizador e o xestor de todo o que
se fai co obxectivo é facilitar o uso do ordenador.

Os sistemas operativos actuais teñen encomendadas moitas misións,
algunhas son familiares para nós e seguro que todos as usamos a cotío

\_\_ Simplificar a relación co usuario:\_\_

\begin{quote}
Sexa cunha interfaz modo texto ou modo gráfico.
\end{quote}

\_\_ Controlar a execución dos programas:\_\_

\begin{quote}
Aceptar os traballos, administrar o xeito no que se realizan, asignar
recursos e finalízalos cando cómpra.
\end{quote}

\_\_ Xestionar os sistemas de arquivo:\_\_

\begin{quote}
manter a lista de arquivos do disco e favorecer a súa organización (por
exemplo en directorios) e a súa manipulación (creación, modificación,
eliminación, etc)
\end{quote}

\_\_ Administrar periféricos\_\_

\begin{quote}
Coordinar e organizar os dispositivos conectados ó ordenador. Con que eu
faga \texttt{Arquivo/Imprimir} podo pasar a papel os meus documentos sen
preocuparme do funcionamento dos rodillos da impresora.
\end{quote}

Outras funcións teñen un carácter máis técnico e serán as que
traballaremos máis polo miúdo:

\_\_ Xestión de permisos e usuarios:\_\_

\begin{quote}
Adxudica permisos de acceso e evita que as accións dun usuario afecten ó
traballo que fai outro. Ou que un usuario cotillee nos documentos de
outro sen permiso.
\end{quote}

\_\_ Control de concurrencia:\_\_

\begin{quote}
Establece prioridades cando se precise usar un recurso. Se varios
programas teñen que usar o procesador non pode ser que o usen todos á
vez e que se mesturen os datos.
\end{quote}

\_\_ Administración de memoria:\_\_

\begin{quote}
Asigna posicións de memoria e xestiona o seu uso. Non necesito
preocuparme das posicións de memoria que estou ocupando.
\end{quote}

\_\_ Control de seguridade:\_\_

\begin{quote}
Garantiza que a información se almacene dun xeito seguro. Uns datos non
poden pisar ós outros.
\end{quote}

\_\_ Apoio a programas:\_\_

\begin{quote}
permitindo o uso de servizo dispoñibles ou chamadas ó sistema.
\end{quote}

\_\_ Control de erros:\_\_

\begin{quote}
Xestiona os erros de hardware e a perda de datos.
\end{quote}
