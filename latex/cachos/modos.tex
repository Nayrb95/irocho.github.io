Se por calquera razón un programa precisa o uso do hardware directamente pódese inserir unhas liñas de código para acceder a funcións concretas que se denominan  \textit{chamadas ó sistema}. Son os fabricantes os que proporcionan esa información en bibliotecas chamadas API. Se quero programar un videoxogo pode ser que teña que acceder á API de Windows e empregar as funcións que me proporciona Microsoft. En principio a relación co hardware só é responsabilidade do sistema operativo.

Para que o programador non se teña que ocupar dos detalles do hardware e non se poda estragar nada,  a maioría das computadoras teñen dous modos de operación: modo kernel e modo usuario.


\begin{itemize}
\item 
O sistema operativo é a peza fundamental do software e  execútase en modo kernel (ou modo supervisor). Neste modo tense acceso completo a todo o hardware e pódese executar calquera instrución que a máquina sexa capaz.
\item O resto do software  execútase en modo usuario: só un subconxunto de instrucións están permitidas. Están prohibidas para os programas que se executan neste modo as que afectan especialmento ó control da máquina ou as que se encargan da entrada e saída de información 
\end{itemize}

