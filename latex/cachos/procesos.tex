Un proceso é un programa en execución. Cada proceso componse do código que se executa e a correspondente estructura de datos. Ambos estarán cargados en memoria e terán uns recursos asignados: espacio en memoria, uso da CPU, etc.  O sistema operativo é o encargado de controlar a execución.\\


O contido da estructura de datos dun proceso que  permite controlar todos os aspectos da súa execución é:
\begin{singlespace}
\begin{description}
	\item[Estado actual do proceso:] Pode estar en execución, agardando, parado,..

	\item[Identificación:] Os procesos teñen cadanseu PID ou sexa un número que permite que o sistema operativo poda identificalo. 
	\item[Prioridade:] Número que indica a vez para a súa execución. O que teña maior prioridade dos que están agardando executarase antes.
	\item[Zona de memoria:] Cada proceso ten reservado un espacio en memoria que non pode ser ocupado por outros procesos.
	\item[Recursos asociados:] Un proceso ten necesidades propias que ten que coñecer o sistema operativo, por exemplo o acceso a un ficheiro  determinado.
\end{description} 
\end{singlespace}

Un proceso pódese crear, executar, poñelo en espera ou matalo. Existen uns procesos que se crean no arranque do sistema eqeu permanecen en segundo plano e son os que están pendentes do correo electrónico, de que se imprima correctamente ou de avisar de eventos da axenda. Estes procesos en Linux chámanse \textit{demos}. Se quixéramos crear a man un proceso neste sistema operativo temos o comando \texttt{fork} e para monitorizar os procesos que se están executando usaremos \texttt{top, ps}. Se queremos rematar un proceso empregaremos \texttt{kill} indicando o PID. Moito olliño con facelo sen estar seguros do que facemos.\\
Cada proceso ocupa un espazo propio en memoria. Con frecuencia é conveniente ter varios fíos de control no mesmo espazo de direccións de memoria que comparten os datos e  que se executan á vez. Desenvolvemos así varias actividades conxuntamente e algunhas  pódense bloquear sen necesidade de bloquear todo o proceso. Descompoñer unha aplicación en varios fíos  que se executan casi en paralelo mellor a eficiencia do sistema. Por exemplo se temos varios procesadores cada un podería executar un fío sen ter que agardar que se execute un tras outro.\\
